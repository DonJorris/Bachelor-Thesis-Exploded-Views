\iflanguage{ngerman}
{\chapter{Fazit und Ausblick}}
{\chapter{Conclusion and Further Work}}

\label{sec:conclusion}
% Anzeigen von mehr oberflächeneigenschaften mittels shadern

In order to solve the problem of occlusion, which occurs in the considered data set, first an analysis of related work was performed and four methods were derived and presented. These were then compared, tested and evaluated by two experts. Furthermore, it was investigated how explosion views can be combined with time-varying data sets and which possibilities arise in this context. The methods presented for creating explosion views were described by the experts as effective and target-oriented. Thus, the goal of avoiding occlusion was achieved. Context preservation is also given with the line method. In addition, a prototype was developed which visualizes the data set and its individual time steps and allows them to be explored interactively in a virtual environment. Here, the methods can be tested and their parameters changed via a UI panel. Subsequently, the prototype was tested and evaluated by a test group. This also showed that apart from some usability improvements, the implemented methods are promising for solving the occlusion problem. Also, the performance of the prototype was analyzed and suggestions for improvement were discussed.

Evaluating the methods revealed additional possibilities and improvements that can be taken up in \textbf{future work}. Among the improvements mentioned in the evaluation chapter, the temporal change of the cell surfaces between the time steps could be morphed. This would improve the still jagged transition between time steps and could provide a more linear time progression that is easier to inspect. Another possibility, which was suggested by the experts, would be to reduce the cell complexity by using NURBS.
This could further reduce the visual complexity of the data set, making it easier to analyze. This could also be achieved by applying the Marching Cubes algorithm and would additionally remove the vertices inside the cells, which could lead to a significant performance increase. An advantage of this would also be that the object complexity could be dynamically adjusted.
Furthermore, it should be investigated how additional cell properties could be visualized in a meaningful way without adding too much visual complexity. There are existing properties in the data set that could be used for this task. It was also suggested to give each individual cell of a population a different shading to make them more distinguishable. 
Similarly, the explosion views could be abandoned and other methods of occlusion avoidance could be tested. Here, for example, the use of the magic lenses described in the related work chapter would be a possibility. A combination of methods might also be feasible.
Overall, however, the work has yielded promising results and suggested new approaches to interactive exploration of data sets, which can be taken up further.