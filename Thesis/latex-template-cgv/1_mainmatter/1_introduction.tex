\iflanguage{ngerman}
{\chapter{Einleitung}}
{\chapter{Introduction}}

\label{sec:introduction}
The chapter should broadly contextualize your research and motivate your work.

\section{Motivation}
Here you motivate, why you are doing your research.

Which problem does this thesis try to solve and how? What is new? Why is it relevant? What is the strategy to solve this problem? Which parts of the problem are  being worked on, which are excluded? 
Why is occlusion a problem, when viewing models and datasets in general? 
Why is the virtual reality a good way to visualize data? What does it offer compared to the traditional desktop screens?
Why is the inspection of surfaces in the virtual space advantageous?

\section{Goal}
This section should clarify, what should be achieved by the work.
How does my implementation solve the occlusion problem?
How do I plan on implementing an exploded view in VR? What is the overall goal for this thesis?
Which questions and task am I trying to answer/solve?

\section{Structure of the Work}
Here the structure can be \emph{briefly} explained.
Possible structure:
- related work, maybe do it chronologically  like Li et al. 
- background/basic concepts/important attributes of exploded views should have
- describe different implementation concepts and categorize them
- explain how I implemented those concepts in my work, which technologies did I use and why, what did I change and why and what did I do that is somewhat new
- Show results of my work/what went right and what did not work
- discuss results, what went well, what was interesting about my implementation and why? Why did some things not work and what could be improved and how? 
- Summarize results and draw a conclusion of the work 
- give an outlook on what could be done to improve the work


