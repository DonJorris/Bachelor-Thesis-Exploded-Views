
\course{Computer Science}
\matriculationyear{2019}
\issuedate{04.10.2022}
\duedate{19.12.2022}
\professor{Prof. Dr. Stefan Gumhold}


\begin{task}[] % If needed, a custom title can be provided between the brackets
	\minisec{\objectivesname}\smallskip	
	
Discrete simulation models for the formation of cellular tissue has matured to a state where large
datasets with high spatial and temporal resolution can be generated based on a description of the
simulation model. Inspection of this data is difficult due to the dense packing of cells leading to
severe occlusion problems. Especially, for surface properties standard techniques like slicing does
not help. The goal of this thesis is to design and implement a VR-based visual analysis tool that
exploits and potentially extends known occlusion avoidance techniques like X-Ray views or explosion views to provide visual access to cell surfaces.
	
	\minisec{\focusname}\smallskip
	\begin{itemize}
		\item Literature research on occlusion avoidance in real-time setting with static and dynamic 3D environments as well as corresponding interaction design.
		\item Concept design for interactive cell complex exploration in VR with direct interaction and time control.
		\item Import of Dataset provided by collaboration partners and preprocessing necessary for real-time interaction
		\item Concept for implementation including informed selection of VR framework
		\item Prototypical implementation of immersive exploration concept
		\item Evaluation of performance and with expert feedback
	\end{itemize}
	\minisec{Optional tasks}\smallskip
	\begin{itemize}
	\item Visualization of attributes on cell surfaces
	\item User study to determine usability score
\end{itemize}
\end{task}

